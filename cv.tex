% LaTeX Curriculum Vitae Template
%
% Copyright (C) 2004-2009 Jason Blevins <jrblevin@sdf.lonestar.org>
% http://jblevins.org/projects/cv-template/
%
% You may use use this document as a template to create your own CV
% and you may redistribute the source code freely. No attribution is
% required in any resulting documents. I do ask that you please leave
% this notice and the above URL in the source code if you choose to
% redistribute this file.

\documentclass[letterpaper]{article}

\usepackage{hyperref}
\usepackage{geometry}
\usepackage[utf8]{inputenc}
\usepackage[T1]{fontenc}
\usepackage[sc,osf]{mathpazo}
\usepackage{tabularx}

\def\name{Aleksander Balicki}
\hypersetup{
  colorlinks = true,
  urlcolor = blue,
  pdfauthor = {\name},
  pdfkeywords = {computer science},
  pdftitle = {\name: CV},
  pdfsubject = CV,
  pdfpagemode = UseNone
}

\geometry{
  body={6.5in, 8.5in},
  left=0.61in,
  right=0.61in,
  top=0.61in,
  bottom=0.61in
}

\pagestyle{myheadings}
\markright{\name}
\thispagestyle{empty}
\usepackage{sectsty}
\sectionfont{\rmfamily\mdseries\Large}
\subsectionfont{\rmfamily\mdseries\itshape\large}
\setlength\parindent{0em}
\renewenvironment{itemize}{
  \begin{list}{}{
    \setlength{\leftmargin}{1.5em}
  }
}{
  \end{list}
}

\begin{document}

\centerline{\huge \name}
\vspace{0.25in}

I'm a {\bf Lead iOS Developer} with a little under 3 years professional experience. I've done complex front-end ({\bf CALayer anchorPoint}
 and {\bf transform} fun), back-end (parallel {\bf NSOperationQueue} processing) and build system tasks (full command line automation of App Store submisson). 

I use {\bf AppCode} instead of Xcode, because it makes coding less mundane. I prefer to use {\bf CocoaPods}, which became a necessity with the number of Apps in Wikia. I always have {\bf Dash} (documentation browser) installed on all my iDevices. I often read Apple manuals at home for fun.

\vspace{0.25in}

\begin{tabularx}{\textwidth}{lXlll}
        Phone: 	& \href{tel:+48695198977}{+48 695 198 977} & Blog: & \href{http://alistra.ghost.io/}{\tt http://alistra.ghost.io/} \\
        Email: 	& \href{mailto:balicki.aleksander@gmail.com}{\tt balicki.aleksander@gmail.com}
        & Github: & \href{http://github.com/alistra/}{\tt http://github.com/alistra/}\\
\end{tabularx}

\section*{Employment}

\begin{itemize}
    \item \href{http://www.wikia.com}{Wikia}, 2013-2015, {\bf iOS Team Tech Lead}, 1 year 4 months, current.
        \begin{itemize}
            \item Developing features and bug fixes for \href{https://itunes.apple.com/us/app/wikia-game-guides-walkthroughs/id422467074?mt=8}{Game Guides} and \href{https://itunes.apple.com/us/artist/wikia-inc./id422467077}{Community Apps}
            \item Second iOS developer in the company, big impact on the architecture and design of the apps
            \item Working on scaling the environment and the code to handle hundreds of \href{https://itunes.apple.com/us/artist/wikia-inc./id422467077}{Community Apps}
            \item Setup of the build process of Wikia iOS apps: {\bf Jenkins}, {\bf Testflight}, {\bf JIRA API}, {\bf Handsfree Automation}
            \item Developing a modularization framework for iOS and implementing it in Wikia apps
            \item Implementing {\bf Analytics}, {\bf i18n}, {\bf Local and Push Notifications} and {\bf A/B Tests} in the apps
        \end{itemize}
    \item \href{http://www.morriscooke.com}{MorrisCooke}, 2012-2013, {\bf Lead iOS developer}, 11 months.
		\begin{itemize}
            \item Lead developer of \href{https://itunes.apple.com/us/app/stick-around-by-tony-vincent/id557949353?mt=8}{StickAround}, team of 4 people
            \item Implemented a parallel, responsive engine for path crossing detection on iOS
            \item Enhanced the presentation-to-video compression time by a factor of 2 in \href{https://itunes.apple.com/us/app/explain-everything/id431493086?mt=8}{Explain Everything}
            \item Working with a large codebase (500k lines of code)
            \item Advanced and complex {\bf UIView} and {\bf CALayer} management and {\bf animations}
            \item Teaching junior developers about the platform and good software design
            \item Introduced a proper bug tracker in the company, code style guidelines.
            \item Architecture of the slide grid in \href{https://itunes.apple.com/us/app/final-argument/id480232096?mt=8}{Final Argument}
            \item Ported Organizer+ to the new {\bf iPhone 5 resolution} and {\bf iOS 6}
		\end{itemize}

	\item Timepie (never released), 2012, {\bf AppleScript developer}, 1 month

    \item \href{http://intres.com.pl}{Intres}, 2012, {\bf iOS and OS X developer} 10 months.
		\begin{itemize}

            \item StoreQ - Web-based file hosting (Dropbox-like) service
                \begin{itemize}
                    \item Created and designed the client apps for both iOS and OS X
                    \item Frameworks used: {\bf fmdb}, {\bf Sparkle}, {\bf CDEvents}, {\bf Growl}
                \end{itemize}

            \item Integrated {\bf iAd} and {\bf AdMob} into \href{http://dguide.pl}{dGuide}

            \item Optimized the amount of data \href{http://dguide.pl}{dGuide} downloads upon startup (from 20MB down to 100KB)

        \end{itemize}

\end{itemize}

\section*{Education}

\begin{itemize}
  	\item Computer Science, University of Wrocław, Master's Degree, 2014.
  	\item Computer Science, University of Wrocław, Bachelor's Degree, 2011.
\end{itemize}

\section*{Skills}

\begin{itemize}
	\item iOS Development \begin{itemize}

		\item {\bf Front end} I know how to create multitouch {\bf UIKit} animations that don't snap, when you add or remove fingers. I have the power to control the time (just in {\bf Core Animation} objects implementing {\bf CAMediaTiming}). I do all the layout in code instead of .xib files, as it's safer and easier to code review. Recently I started using {\bf AutoLayout}, but most of my complex layout work was done by hand in layoutSubviews. I always pester the designers, that something is not according to the {\bf Apple Human Interface Guidelines}.
		
		\item {\bf Back end} I don't understand why people use {\bf AFNetworking} if they don't cancel the {\bf NSOperations}, they could have just used the Apple libraries. I learned the hard way, that besides using a {\bf Core Data} managed object context on only one queue, you also have to intialize it on the same queue. During code reviews I usually have to explain what a {\bf dispatch\_group} does and that doing a lot of boolean flags isn't the way. I wanted to add {\bf Swift} to the app, but {\bf CocoaPods} not yet support Swift fully.
		
		\item {\bf Testing} I've written unit tests, but most of my testing was done in {\bf Frank}, selenium-like ruby acceptance test framework. I wrote support for resetting the state of the simulator and helped the QA team add accessibilityLabels. I've helped create a Frank scenario to generate all the required screenshots in the Apps, because doing 5 screens on 4 devices for 12 languages for 25 apps isn't feasible manually.
		
		\item {\bf Build system} Building 125+ apps every time you develop something is easier when you have full automation. I developed a build system that works from the command line, supports multiple Apple programs (including {\bf Enterprise}), signs the IPA properly and uploads to {\bf Testflight} or the {\bf iTunes Connect}. I use {\bf xctool} as the reporters nice and easily produce {\bf oclint} required files. Some scripts required {\bf AppleScript}, as not everything has a command-line equivalent.

	\end{itemize}	

	\item Languages - Objective-C, Haskell, Python, Ruby, C, able to learn new programming languages fast.
	\item Knowledge of algorithms, data structures, databases and networking.
    \item Linux/Unix command line tools (vim, git) and scripting, network and Linux administration.
	\item Contributed small patches to open source projects: Stash, xxxterm, jumanji, rust;\\
		 haskell libraries: alex, imgurder, derive, Safe, XMonad.Util.Paste
	\item Interested in compilers, static code analysis, development tools, OS X, iOS and Linux.
\end{itemize}

\section*{Projects}
\begin{itemize}
    \item imgur-directory-listing - a photoblog, that uploads the pictures to \href{http://imgur.com}{imgur},
        designed for home servers with a slow upload connections,\\ initially written in {\bf Ruby on Rails}, rewritten in {\bf Yesod}.
\end{itemize}

\bigskip

% Footer
\begin{center}
  \begin{footnotesize}
    Last updated: \today
    \hfill
    \href{http://s3.amazonaws.com/alistra-cv/current/cv.pdf}
    {\texttt{http://s3.amazonaws.com/alistra-cv/current/cv.pdf}}
  \end{footnotesize}
\end{center}

\end{document}
