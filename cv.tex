% LaTeX Curriculum Vitae Template
%
% Copyright (C) 2004-2009 Jason Blevins <jrblevin@sdf.lonestar.org>
% http://jblevins.org/projects/cv-template/
%
% You may use use this document as a template to create your own CV
% and you may redistribute the source code freely. No attribution is
% required in any resulting documents. I do ask that you please leave
% this notice and the above URL in the source code if you choose to
% redistribute this file.

\documentclass[letterpaper]{article}

\usepackage{hyperref}
\usepackage{geometry}
\usepackage[utf8]{inputenc}
\usepackage[T1]{fontenc}
\usepackage[sc,osf]{mathpazo}
\usepackage{tabularx}

\def\name{Aleksander Balicki}

% Replace this with a link to your CV if you like, or set it empty
% (as in \def\footerlink{}) to remove the link in the footer:

% The following metadata will show up in the PDF properties
\hypersetup{
  colorlinks = true,
  urlcolor = black,
  pdfauthor = {\name},
  pdfkeywords = {computer science},
  pdftitle = {\name: Curriculum Vitae},
  pdfsubject = {Curriculum Vitae},
  pdfpagemode = UseNone
}

\geometry{
  body={6.5in, 8.5in},
  left=0.61in,
  right=0.61in,
  top=0.61in,
  bottom=0.61in
}

% Customize page headers
\pagestyle{myheadings}
\markright{\name}
\thispagestyle{empty}

% Custom section fonts
\usepackage{sectsty}
\sectionfont{\rmfamily\mdseries\Large}
\subsectionfont{\rmfamily\mdseries\itshape\large}

% Other possible font commands include:
% \ttfamily for teletype,
% \sffamily for sans serif,
% \bfseries for bold,
% \scshape for small caps,
% \normalsize, \large, \Large, \LARGE sizes.

% Don't indent paragraphs.
\setlength\parindent{0em}

% Make lists without bullets
\renewenvironment{itemize}{
  \begin{list}{}{
    \setlength{\leftmargin}{1.5em}
  }
}{
  \end{list}
}

\begin{document}

\centerline{\huge \name}
%{\huge \name}

\vspace{0.25in}

\begin{tabularx}{\textwidth}{lXlll}
        Phone: 	& +48 695 198 977 & Homepage: & \href{http://alistra.ath.cx/}{\tt http://alistra.ath.cx/} \\
        Email: 	& \href{mailto:balicki.aleksander@gmail.com}{\tt balicki.aleksander@gmail.com}
        & Github: & \href{http://github.com/alistra/}{\tt http://github.com/alistra/}\\
    \end{tabularx}

\section*{Employment}

\begin{itemize}
    \item \href{http://www.morriscooke.com}{MorrisCooke}, 2012, {\bf iOS developer} 1 month, current.
		\begin{itemize}

            \item Ported Organizer+ to the new {\bf iPhone 5 resolution} and {\bf iOS 6}

		\end{itemize}

	\item Stealth-mode startup, 2012, {\bf OS X developer} 1 month
		\begin{itemize}

			\item Interacting with major web browsers (Firefox, Chrome, Safari) using {\bf AppleScript}

		\end{itemize}

    \item \href{http://intres.com.pl}{Intres}, 2012, {\bf iOS and OS X developer} 10 months.
		\begin{itemize}
            \item \href{http://storeq.com}{StoreQ} - Web-based file hosting service
                \begin{itemize}
                    \item Created and designed the apps for both iOS and OS X
                    \item Patched a bug in the {\bf SudzC} library involving too small integer types being generated for the WSDL type
                    \item Frameworks used: {\bf fmdb}, {\bf Sparkle}, {\bf CDEvents}, {\bf Growl}
                \end{itemize}

            \item Added {\bf iAd} and {\bf AdMob} frameworks to \href{http://dguide.pl}{dGuide}

            \item Optimized the amount of data \href{http://dguide.pl}{dGuide} downloads upon startup (from 20MB down to 100KB),\\
                required modifying the {\bf Django} backend

    \end{itemize}

    \item \href{http://www.linkedin.com/company/synthcomm-sp.-z-o.o.}{Synthcomm}, 2010, {\bf iPhone developer} internship, 2 months.
		\begin{itemize}

			\item Created and designed an app for syncing address book photos,
                using {\bf Facebook API}, {\bf Ruby on rails} backend.

		\end{itemize}

    \item \href{http://power.com.pl}{Power Media}, 2008, {\bf Linux system administrator} internship, 3 months.
	    \begin{itemize}

		\item Created a linux ISO, that logged the workstation hardware to the destined email address upon startup

		\item Wrote a ruby script that visualized the progress of Bacula backups in ASCII art

	    \end{itemize}
\end{itemize}

\section*{Projects}
\begin{itemize}
    \item imgur-directory-listing - a photoblog, that uploads the pictures to \href{http://imgur.com}{imgur},
        designed for slow upload connections,\\ initially written in {\bf Ruby on Rails}, rewritten in {\bf Yesod}.
\end{itemize}

\section*{Education}

\begin{itemize}
  	\item Computer Science, University of Wrocław, master's program, planning to finish 2012.
  	\item Computer Science, University of Wrocław, finished the Bachelor's Degree on July 2011.
\end{itemize}

\section*{Skills}

\begin{itemize}
	\item Objective-C, Haskell, Python, C and I'm able to learn new programming languages fast.
	\item Knowledge of Cocoa[Touch], GCD, iOS and OS X HIG, AppleScript basics.
	\item Knowledge of algorithms, data structures, databases and networking.
    \item Linux/Unix command line tools (vim, git) and scripting, network and Linux administration.
	\item Contributed small patches to open source projects: xxxterm, jumanji, rust;\\
		 haskell libraries: alex, imgurder, derive, Safe, XMonad.Util.Paste
	\item Interested in compilers, static code analysis, computational complexity and Linux.
\end{itemize}

\bigskip

% Footer
\begin{center}
  \begin{footnotesize}
    Last updated: \today \hfill \href{http://alistra.ath.cx/cv.pdf}{\texttt{http://alistra.ath.cx/cv.pdf}}
  \end{footnotesize}
\end{center}

\end{document}
